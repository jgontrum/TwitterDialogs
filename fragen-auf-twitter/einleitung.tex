\documentclass[main.tex]{subfiles} 

\begin{document}

\section{Einleitung}

Twitter ist ein Echtzeitanwendung, die es dem Nutzer erlaubt, von unterschiedlichen Plattformen (Desktop-Computer, Smartphone) maximal 140 Zeichen lange Kurznachrichten zu versenden. Dabei folgt der Dienst gerade nicht dem klassischen Sender-Empfänger-Paradigma der Sprachwissenschaft, sondern eine Nachricht wird für jeden zugänglich der Öffentlichkeit präsentiert. Diese Form der Kommunikation erlaubt dem Sender einer Nachricht ein riesiges Publikum zu erreichen aber auch gänzlich ohne Empfänger zu enden. 
Interessant für die Sprachwissenschaft ist Twitter aufgrund seines informellen Charakters. Bisher greift die klassische Korpuslinguistik vornehmlich auf Quellen zurück, die sich durch einen gewissenhaften und regelkonformen Umgang mit Sprache auszeichnen (Zeitungen, Parlamentsreden o.ä.). Twitterdaten hingegen zeichnen sich durch einen informellen Schreibstil aus: Benutzer des Dienstes gehen kreativ mit Sprache um, passen sie den technischen Gegebenheiten an und legen oftmals keinen Wert auf formale Regeln. Der soziale Charakter des Mediums führt oftmals dazu, dass bewusst mit Normabweichungen gespielt wird. Gerade hierin besteht für den Sprachwissenschaftler eine große Möglichkeit, synchrone Sprachwissenschaft zu betreiben und Phänomene zu entdecken.

In der vorliegenden Arbeit betrachten wir einen Aspekt von Twitter-Kommunikation: Fragen in Twitter-Dialogen. Twitter erlaubt eine besondere Form der dialogischen Kommunikation: durch \textit{Replies} können auf bestimmte Nachrichten Antworten gesendet werden. Durch Analyse von Metadaten können diese Dialoge vollumfänglich rekonstruiert werden.

Uns interessierte, welche Rolle Fragen auf Twitter überhaupt spielen. Wie oft werden Fragen beantwortet? Gibt es Muster der Beantwortung von Fragen? Wenn überhaupt, unterscheiden sich Fragestellungen auf Twitter von normalsprachlichen Fragen.

Um diese Fragen beantworten zu können, war es zunächst notwendig, Dialogstrukturen zu erkennen und durchsuchbar zu machen. In einem weiteren Schritt mussten Fragen erkannt werden.

\end{document}